\subsection{Analysis of Problem}
To work out the four problems, our solutions will be proceeded as follows.
\begin{itemize}
  \item[$\circledcirc$] \textbf{Problem 1:} In the nature, the process of fungi decomposing lignin or cellulose involves complex biochemical reactions. Hence, to simplify the problem, we are going to divide the decomposition situations into two types: ideal cylinder and plane. Based on that, we could conduct a symmetrization operation on decomposition environment of the fungi to facilitate the connection of this process with the \textbf{Gauss theorem} and Gauss surface in electromagnetic. Meanwhile, \textbf{fungal activity factor} (ACT) will be added to the model, which is determined by the suitable living concentration of specific fungal types, moisture tolerance and the influence between different fungal species.
  \item[$\circledcirc$] \textbf{Problem 2:} Based on Figure 1 and Figure 2 given in problem sheet, we plan to perform a function fitting derivation, and then carry out the fungi selection operation, which simplifies the overall model building process, and obtains more practical data. When considering two fungal populations living together, due to competition for habitat and food, the interaction between them would hinder each other's growth and reproduction. Therefore, we are going to introduce the \textbf{Relative Growth Blocking Index} (RGB) on the basis of Single-group Logistic Model, and establish \textbf{Multi-groups Logistic Model} to predict trends of fungal populations in short term and long term.
  \item[$\circledcirc$] \textbf{Problem 3:} To analyze different influences on each isolate and fungal combinations caused by different environments. We will conduct two predictions for single fungal species and combinations of fungal species, respectively. \textbf{Logistic Model} and \textbf{Gray Prediction} will be used in the prediction process. Meanwhile, five environmental conditions, including arid, semi-arid, temperate, arboreal, and tropical rain forests, will be candidates for the optimal environmental condition.
  \item[$\circledcirc$] \textbf{Problem 4:} We will carry out an ecologically significant analysis of the impact of fungal population biodiversity on the \textbf{carbon cycle} in authentic nature from the perspective of the ecosystem, and write the important conclusions in the \textbf{\textit{article for college level biology textbook}}.
\end{itemize}
\par
Above is a sketch of the analysis process on the four problems.