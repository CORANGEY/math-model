\subsection{Fungi Selection}
There are many types of fungi in the nature, and the selection of specific fungal species for research would be a complicated problem. Therefore, with \textit{Figure~1} and \textit{Figure~2} in the problem sheet, we fit the three functions corresponding with three curves of \textit{hyphal extension rate $(mm/day)$ - decomposition rate ($\%$ mass loss over 122 days)} under conditions of $22^\circ C$, $16^\circ C$ and $10^\circ C$. We also fit the linear function corresponding with the \textit{relative moisture tolerance - decomposition rate curve} under relative scaling. The fitting functions are listed in \textit{Eq.~(\ref{10eq})} and \textit{Eq.~(\ref{11eq})} as follows.
\begin{equation}
  \label{10eq}
  \begin{cases}
    DR_{22^\circ C} = 5.46 \ln (HER) + 16.95 + \varepsilon_1 \\
    DR_{16^\circ C} = 4.94 \ln (HER) + 8.31 + \varepsilon_2  \\
    DR_{10^\circ C} = 2.22 \ln (HER) + 6.03 + \varepsilon_3
  \end{cases}
\end{equation}
where $\varepsilon_i (i=1,\ 2,\ 3) \sim N(0, \sigma^2)$, $HER$ represents the \textbf{hyphal extension rate} and $DR$ represents the \textbf{decomposition rate}.
\begin{equation}
  \label{11eq}
  \log (DR) = 0.995 RMT +1.882
\end{equation}
where $RMT$ represents the relative moisture tolerance. In the regression fitting process, the sample correlation coefficient is $R^2$~\cite{R2}, the results of which are listed as follows.
\begin{table}[H]
  \centering
  \caption{Sample correlation coefficient results.}
  \label{samplecorrelation}
  \begin{tabular*}{\hsize}{@{\extracolsep{\fill}}cccc}
    \toprule
    & Temperature ($^\circ C$) & $R^2$ & \\
    \midrule
    & $22$ & $0.1662$ & \\
    & $16$ & $0.3921$ & \\
    & $10$ & $0.2504$ & \\
    \bottomrule
  \end{tabular*}
\end{table}
Since $R^2$ does not significantly tend to zero, three random disturbance terms are added to improve the accuracy of the fitting.
\par
In order to obtain representative fungal species, we take temperature, moisture and hyphal extension rate into account in the \textbf{fungi selection}. Finally, select five typical fungal species as follows.
\begin{table}[H]
  \centering
  \caption{Five selected typical fungal species.}
  \label{fivetypicalfungalspecies}
  \begin{tabular*}{\hsize}{@{\extracolsep{\fill}}cccc}
    \toprule
    Fungi & Optimal Temperature ($^\circ C$) & Relative Moisture & $HER$ ($mm/day$) \\
    \midrule
    Fungi A ($F_A$) & $22$ & $-1.0$ & $5$ \\
    Fungi B ($F_B$) & $16$ & $-0.5$ & $5$ \\
    Fungi C ($F_C$) & $10$ & $0$ & $5$ \\
    Fungi D ($F_D$) & $16$ & $0.5$ & $5$ \\
    Fungi E ($F_E$) & $22$ & $1.0$ & $5$ \\
    \bottomrule
  \end{tabular*}
\end{table}
Substitute the data in the above table into \textit{Eq.~(\ref{10eq})} and \textit{Eq.~(\ref{11eq})} and perform logarithmic process. We obtain the hyphal extension rate of five fungal species ($F_A\sim F_E$) as $5\ mm/day$. The corresponding decomposition rates in different environmental conditions are listed as follows.
\begin{table}[H]
  \centering
  \caption{Decomposition rates in different environmental conditions.}
  \label{decompositionrates}
  \begin{tabular*}{\hsize}{@{\extracolsep{\fill}}cccc}
    \toprule
    & Fungi & Decomposition Rate ($\%$) & \\
    \midrule
    & $F_A$ & $13.32$ & \\
    & $F_B$ & $8.83$ & \\
    & $F_C$ & $5.74$ & \\
    & $F_D$ & $9.32$ & \\
    & $F_E$ & $14.31$ & \\
    \bottomrule
  \end{tabular*}
\end{table}
\textit{Table~\ref{fivetypicalfungalspecies}} and \textit{Table~\ref{decompositionrates}} provides data support for research on the interactions of different fungal species, which makes it convenient for the modeling and analysis later.